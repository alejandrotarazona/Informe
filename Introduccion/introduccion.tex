	\chapter*{INTRODUCCIÓN}
    \addcontentsline{toc}{chapter}{INTRODUCCIÓN}
    
    Las nuevas tecnologías han permitido que la información llegue a más personas en mucho menos tiempo, ha empezado incluso a cambiar la forma en que nos relacionamos con nuestros semejantes. Programas que facilitan las labores de oficina, planificación, cálculos y presentaciones, ya sea de forma individual o colaborativa, correo, comercio y banca electrónica, redes sociales; son cosas cotidianas hoy en día.
    
    Tenemos también personas desfavorecidas que, por falta de recursos, tiempo o recursos, no han tenido acceso a las innovaciones tecnológicas, viendose así obstaculizadas en su acceso a la información de los respectivos representates u organizaciones que son de su interés.
    
    El siguiente informe se enfoca en el proyecto realizado por profesores y alumnos de la Universidad Simón Bolívar, en el marco del trabajo de servicio comunitario, para facilitar el acceso a las nuevas tecnologías de información a la comunidad obrera de dicha universidad con el fin de mejorar su calidad de vida e incluirlos en la era de la tecnología.
    
	\pagebreak