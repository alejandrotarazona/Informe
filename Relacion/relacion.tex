\chapter{Relación del proyecto trabajado con la formación académica del estudiante}

    Durante la carrera es necesario el uso de las diferentes tecnologías de información para la elaboración de documentos e informes (Writer), capturas de datos (Calc) y comunicación (Correo eléctrónico) entre el estudiante y sus compañeros. Con ello se hizo más fácil la comprensión, por parte del estudiante, de las herramientas de vanguardia en los ámbitos necesarios para poder así atender de mejor manera las dudas surgidas durante las clases.
    
    Adicionalmente, la experiencia como preparador académico jugó un papel importante en la organización y el manejo de los alumnos durante el desarrollo del curso permitiendo una apropiada organización y adaptación a los distintos ritmos de aprendizaje que demostraron los alumnos.
    
%    El uso básico del computador es fundamental, respecto a las herramientas de ofimática tenemos lo siguiente:

%    \begin{itemize}
%        \item Writer:
        
%        Usado regularmente para la redacción de informes y documentar los proyectos realizados en el transcurso de la carrera. Se usan las funcionalidades de formato, inclusión de gráficos, imágenes, tablas e hipervículos.
        
%        \item Calc:
        
%        Usado para la recopilación y organización de datos. Se usan las funcionalidades de creación de fórmulas, creación de gráficos y macros.
        
%        \item Impress:
        
%        Usado para la creación de presentaciones durante las exposiciones. Se usan las funcionalidades de inserción de imágenes, tablas, videos, audio, aniamciones y transisiones personalizadas.
%    \end{itemize}
    
%    En cuanto a las herramientas de red, se tienen:
    
%    \begin{itemize}
%        \item Google Drive:
%        Usado para compartir y editar archivos relativos a los proyectos.
%        \item Banca Electrónica:
%        No es de uso exclusivo de la carrera de ingeniría de la computación. Se usa regularmente para el pago de los distintos aranceles de la universidad.
%        \item Redes Sociales:
%        También usados para la coordinación entre los participantes de un proyectos.
%    \end{itemize}
 
\pagebreak