\documentclass[letterpaper,12pt]{article}

\usepackage{polyglossia}
\setdefaultlanguage{spanish}

%\usepackage[utf8]{inputenc}

\usepackage{graphicx}
\usepackage{footnote}
\usepackage[top=2cm, left=2cm, right=2cm, bottom=2cm]{geometry}
\linespread{1.0}
\parindent 3ex 

\DeclareGraphicsExtensions{.jpg,.pdf,.png}

\begin{document}
	
	\begin{center}
		\includegraphics{./img/logo.png} \\
		Universidad Simón Bolívar \\
		Coordinación de Ingeniería de la Computación \\ 
		\vfill
		{\LARGE \textbf{ALFABETIZACIÓN DIGITAL PARA OBREROS \\ (proyecto piloto SUTES /USB)} }
		\vfill
		\underline{\textbf{Elaborado por:}}\\
		Alejandro Tarazona \\
		\underline{\textbf{Tutor Institucional:}}\\
		Soraya Carrasquel \\
		\underline{\textbf{Representante de la Comunidad:}}\\
		José Bermúdez
		\linebreak
        \linebreak
		\underline{\textbf{Fecha de Culminación:}}\\
		
	\end{center}
	\pagebreak
    
	\tableofcontents
	\pagebreak
	
	\section{INTRODUCCIÓN}
	\pagebreak
	
	\section{JUSTIFICACIÓN DEL SERVICIO COMUNITARIO}
		Muchos servicios e información actualmente sólo están disponibles en Internet. Su acceso puede ser crítico. Por ejemplo SUTES ha tenido personas desfavorecidas porque no supieron leer información enviada por Internet sobre lapsos de espera del HCM. Los obreros deben incluirse a la sociedad digital.
        
        Esta inclusión permite no sólo la mejora del manejo de la información sino que además significa una mejora en la calidad de vida de los mismos, permitiéndoles el uso de las herramientas de internet para la adquisición de nuevos conocimientos y la agilización de labores cotidianas.
	\pagebreak
	
	\section{DESCRIPCIÓN DEL PROBLEMA}
		\subsection{DESCRIPCIÓN DE LA COMUNIDAD}
			Comunidad de obreros que laboran en la Universidad Simón Bolívar. Edades comprendidas entre 18 y 60 años.
            
		\subsection{ANTECEDENTES DEL PROYECTO}
	\pagebreak
	
	\section{DESARROLLO DEL PROYECTO}

        \subsection{TÍTULO DEL PROYECTO}
            Alfabetización digital para obreros (proyecto piloto SUTES /USB).
        \subsection{OBJETIVO GENERAL}
            Capacitar al obrero para el uso de tecnologías infromáticas.
            
        \subsection{OBJETIVOS ESPECÍFICOS}
            \begin{enumerate}
                \item Usar navegadores de internet
                \item Usar buscadores en internet
                \item Usar correo electrónico
                \item Usar banca electrónica
                \item Usar redes sociales
                \item Usar oficinas virtuales
                \item Usar responsablemente el internet
                \item Usar herramientas de ofimática
                \item Usar internet en autoaprendizaje
            \end{enumerate}
        \subsection{EJECUCIÓN DE ACTIVIDADES REALIZADAS}
         \begin{enumerate}
             \item \underline{\textbf{Recibir capacitación en educación de adultos.(8 horas)}}
             
             La capacitación fue dictada por la Lic. Evelyn Abdala en 3 reuniones pautadas entre el 08 de diciembre de 2014 y el 14 de enero de 2015 y la presentaci\'{o}n de la primera clase ante los respectivos tutores el d\'{i}a 22 de enero de 2015. Se esperaba que las reuniones fuesen semanales pero dada la disponibilidad de los preparadores y de la Lic. Abdala, esto no pudo llevarse a cabo.
             
             En las reuniones se nos explic\'{o} el c\'{o}mo podemos abordar a una población especial ya que tenemos como objetivo llegarle a personas de 3ra edad (potencialmente) debíamos estar preparados con técnicas que nos permitieran suavizar el cambio y actualizar a nuestros alumnos de la manera más cómoda posible. En este aspecto se nos instruyó con técnicas verbales y didácticas para que los alumnos pudiesen tener el primer contacto con el computador.
             
             Dado que los preparadores ya estabamos familiarizados con la enseñanza, se procedió a orientar nuestras capacidades al nuevo entorno donde, a diferencia de las preparadurías académicas, no podíamos suponer acceso a tecnologías ni conocimientos previos del tema.
             
             La Lic. Abdala hizo incapié en el problema del lenguaje ya que uno como preparador tiende a familiarizarse con los alumnos y eso podría desencadenar una pérdida de control sobre el proceso de aprendizaje.
             
             Finalmente se nos enseñó a manejar las diferentes velocidades de aprendizaje en los grupos y el cómo mantener el orden en los mismos sin llegar a desalentar a ninguno.
             
             \item \underline{\textbf{Realizar entrevistas a los obreros.(12 horas)}}
             
             Las entrevistas fueron realizadas en las instalaciones de SUTES desde el día 12 de enero de 2015 hasta el 19 de enero de 2015, en lapsos conjuntos no mayores de 4 horas por preparador por día.
             
             En esta fase procedimos a recabar información sobre los aspirantes al curso en los ámbitos socio-económico y académico, así como aspiraciones e intereses personales de los mismos.
             
             \item \underline{\textbf{Hacer diagnostico personal de cada participante.}}
             \item \underline{\textbf{Elaborar planes de capacitación.}}
             \item \underline{\textbf{Preparar material de apoyo.}}
             \item \underline{\textbf{Dictar clases en cursos de capacitación.}}
             \item \underline{\textbf{Acompañar en el proceso de aprendizaje.}}
             \item \underline{\textbf{Validar logros de los participantes.}}
             \item \underline{\textbf{Elaboración de Informe y Certificados.}}
            \end{enumerate}
	\pagebreak
	
	\section{RELACIÓN DEL PROYECTO TRABAJADO CON LA FORMACIÓN ACADÉMICA	DEL ESTUDIANTE}
	\pagebreak
	
	\section{CONCLUSIONES Y RECOMENDACIONES}
	\pagebreak
	
	\section{BIBLIOGRAFÍA}
	\pagebreak
	
	\section{ANEXOS}



\end{document}