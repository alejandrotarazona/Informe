\documentclass[letterpaper,12pt]{article}

\usepackage[spanish]{babel}
\usepackage[utf8]{inputenc}
\usepackage{graphicx}
\usepackage{footnote}
\usepackage[top=2cm, left=2cm, right=2cm, bottom=2cm]{geometry}
\linespread{1.0}
\parindent 3ex 

\DeclareGraphicsExtensions{.jpg,.pdf,.png}

\begin{document}
	
	\begin{center}
		\includegraphics{./img/logo.png} \\
		Universidad Simón Bolívar \\
		Coordinación de Ingeniería de la Computación \\ 
		\vfill
		{\LARGE \textbf{ALFABETIZACIÓN DIGITAL PARA OBREROS \\ (proyecto piloto SUTES /USB)} }
		\vfill
		\underline{\textbf{Elaborado por:}}\\
		Alejandro Tarazona \\
		\underline{\textbf{Tutor Institucional:}}\\
		Soraya Carrasquel \\
		\underline{\textbf{Representante de la Comunidad:}}\\
		José Bermúdez
		\linebreak
        \linebreak
		\underline{\textbf{Fecha de Culminación:}}\\
		
	\end{center}
	\pagebreak
    
	\tableofcontents
	\pagebreak
	
	\section{INTRODUCCIÓN}
	\pagebreak
	
	\section{JUSTIFICACIÓN DEL SERVICIO COMUNITARIO}
		Muchos servicios e información actualmente sólo están disponibles en Internet. Su acceso puede ser crítico. Por ejemplo SUTES ha tenido personas desfavorecidas porque no supieron leer información enviada por Internet sobre lapsos de espera del HCM. Los obreros deben incluirse a la sociedad digital.
        
        Esta inclusión permite no sólo la mejora del manejo de la información sino que además significa una mejora en la calidad de vida de los mismos, permitiéndoles el uso de las herramientas de internet para la adquisición de nuevos conocimientos y la agilización de labores cotidianas.
	\pagebreak
	
	\section{DESCRIPCIÓN DEL PROBLEMA}
		\subsection{DESCRIPCIÓN DE LA COMUNIDAD}
			Comunidad de obreros que laboran en la Universidad Simón Bolívar. Edades
			comprendidas entre 18 y 60 años.
		\subsection{ANTECEDENTES DEL PROYECTO}
	\pagebreak
	
	\section{DESARROLLO DEL PROYECTO}
<<<<<<< HEAD
=======
        \subsection{TÍTULO DEL PROYECTO}
            Alfabetización digital para obreros (proyecto piloto SUTES /USB).
        \subsection{OBJETIVO GENERAL}
            Capacitar al obrero para el uso de tecnologías infromáticas.
            
        \subsection{OBJETIVOS ESPECÍFICOS}
            \begin{enumerate}
                \item Usar navegadores de internet
                \item Usar buscadores en internet
                \item Usar correo electrónico
                \item Usar banca electrónica
                \item Usar redes sociales
                \item Usar oficinas virtuales
                \item Usar responsablemente el internet
                \item Usar herramientas de ofimática
                \item Usar internet en autoaprendizaje
            \end{enumerate}
        \subsection{EJECUCIÓN DE ACTIVIDADES REALIZADAS}
>>>>>>> parent of 4fa9ea3... Empezando actividades
	\pagebreak
	
	\section{RELACIÓN DEL PROYECTO TRABAJADO CON LA FORMACIÓN ACADÉMICA	DEL ESTUDIANTE}
	\pagebreak
	
	\section{CONCLUSIONES Y RECOMENDACIONES}
	\pagebreak
	
	\section{BIBLIOGRAFÍA}
	\pagebreak
	
	\section{ANEXOS}



\end{document}