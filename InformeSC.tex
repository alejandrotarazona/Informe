\documentclass[letterpaper,12pt]{article}

\usepackage[english]{babel}
\usepackage[utf8]{inputenc}
\usepackage{graphicx}
\usepackage{footnote}
\usepackage[top=2cm, left=3cm, right=3cm, bottom=2cm]{geometry}
\linespread{1.0}
\parindent 3 ex

\DeclareGraphicsExtensions{.jpg,.pdf,.png}

\begin{document}
	
	\begin{center}
		\includegraphics{./img/logo.png} \\
		Universidad Simón Bolívar \\
		Coordinación de Ingeniería de la Computación
        
		\vfill
		{\LARGE \textbf{ALFABETIZACIÓN DIGITAL PARA OBREROS \\ (proyecto piloto SUTES /USB)} }
		\vfill
        
        
		\underline{Elaborado por:}\\
		Alejandro Tarazona
        \linebreak
        \linebreak
		\underline{Tutor Institucional:}\\
		Soraya Carrasquel
        \linebreak
        \linebreak
		\underline{Representante de la Comunidad:}\\
		José Bermúdez
		\linebreak
        \linebreak
        \linebreak
		\underline{Fecha de Culminación:}\\
		
	\end{center}
	\pagebreak
    
	\tableofcontents
	\pagebreak
	
	\section{INTRODUCCIÓN}
	\pagebreak
	
	\section{JUSTIFICACIÓN DEL SERVICIO\\ COMUNITARIO}
    
		Muchos servicios e información actualmente sólo están disponibles en Internet. Su acceso puede ser crítico. Por ejemplo SUTES ha tenido personas desfavorecidas porque no supieron leer información enviada por Internet sobre lapsos de espera del HCM. Los obreros deben incluirse a la sociedad digital.
        
        Esta inclusión permite no sólo la mejora del manejo de la información sino que además significa una mejora en la calidad de vida de los mismos, permitiéndoles el uso de las herramientas de internet para la adquisición de nuevos conocimientos y la agilización de labores cotidianas.
	\pagebreak
	
	\section{DESCRIPCIÓN DEL PROBLEMA}
		\subsection{DESCRIPCIÓN DE LA COMUNIDAD}
        
			Comunidad de obreros que laboran en la Universidad Simón Bolívar. Edades comprendidas entre 18 y 60 años.
		\subsection{ANTECEDENTES DEL PROYECTO}
	\pagebreak
	
	\section{DESARROLLO DEL PROYECTO}
        \subsection{TÍTULO DEL PROYECTO}
        
            Alfabetización digital para obreros (proyecto piloto SUTES /USB).
        \subsection{OBJETIVO GENERAL}
        
            Capacitar al obrero para el uso de tecnologías infromáticas.
            
        \subsection{OBJETIVOS ESPECÍFICOS}
        
            \begin{enumerate}
                \item Usar navegadores de internet
                \item Usar buscadores en internet
                \item Usar correo electrónico
                \item Usar banca electrónica
                \item Usar redes sociales
                \item Usar oficinas virtuales
                \item Usar responsablemente el internet
                \item Usar herramientas de ofimática
                \item Usar internet en autoaprendizaje
            \end{enumerate}
        \subsection{EJECUCIÓN DE ACTIVIDADES REALIZADAS}
	\pagebreak
	
	\section{RELACIÓN DEL PROYECTO TRABAJADO CON LA FORMACIÓN ACADÉMICA	DEL\\ ESTUDIANTE}
	\pagebreak
	
	\section{CONCLUSIONES}
    Durante la realización del servicio comunitario me tocó tener a dos muchachos que tenían el ánimo para el trabajo y grandes deseos y espectativas sobre el curso pero carecían de los medios para llevar un autoaprendizaje fuera del aula lo cual, considero,
    fue un obstaculo grande dado que la labor de aprendizaje y refuerzo se veía reducida a las horas de clase.
    
    Me pude dar cuenta de la gran necesidad que tienen las personas del conocimiento y uso cotidiano de las herramientas informáticas, el mundo se está automatizando y los más desfavorecidos son aquellos que, por razones diversas, no tienen acceso a las nuevas herramientas de conocimiento.
    
    No es esto una limitante al desenvolvimiento y realización personal sin embargo varios de los entrevistados, y posteriormente alumnos, demostraron tener capacidades que, sin el uso de las herramientas mencionadas, estaban en una situación desfavorable respecto de sus colegas que si tenían acceso y conocimientos sobre ellas.
    
    Por otro lado, me maravilló el potencial que hay y que si es bien trabajado se puede
    lograr que ellos mismos sean autodidactas.    
	\pagebreak
    
    \section{RECOMENDACIONES}
	Como recomendación final, sugiero que se abra un espacio, no necesariamente exclusivo, para el uso de estos conocimientos por parte de los participantes para así poder contar con el refuerzo y la práctica libre que, en caso del uso de navegadores y redes sociales, es fundamental para su futura retención.
    \pagebreak
    
	\section{BIBLIOGRAFÍA}
	\pagebreak
	
	\section{ANEXOS}



\end{document}