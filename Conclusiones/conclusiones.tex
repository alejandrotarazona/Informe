\chapter*{Conclusiones}
\addcontentsline{toc}{chapter}{Conclusiones}
    Se observó la gran necesidad que tienen las personas del conocimiento y uso cotidiano de las herramientas informáticas, el mundo se está automatizando y los más desfavorecidos son aquellos que, por razones diversas, no tienen acceso a las nuevas herramientas de conocimiento.
    
    No es esto una limitante al desenvolvimiento y realización personal sin embargo varios de los entrevistados, y posteriormente alumnos, demostraron tener capacidades que, sin el uso de las herramientas mencionadas, estaban en una situación desfavorable respecto de sus colegas que si tenían acceso y conocimientos sobre ellas.

    Durante la realización del servicio comunitario me tocó tener a dos muchachos que tenían el ánimo para el trabajo y grandes deseos y espectativas sobre el curso pero carecían de los medios para llevar un autoaprendizaje fuera del aula lo cual, considero, fue un obstaculo grande dado que la labor de aprendizaje y refuerzo se veía reducida a las horas de clase.
    
    Por otro lado, me maravilló el potencial que hay y que si es bien trabajado se puede lograr que ellos mismos sean autodidactas.
    
    Por la diversidad de las situaciones socio-económicas, en relación a mis experiencias previas como preparador académico en la USB, tuve que abordar los temas y replantear el cronograma adherido a dichas restricciones, lo cual representó un reto en cuanto al tiempo dedicado a las actividades, las técnicas de aprendizaje y la dinámica de clase utilizada en el curso.
    
	\pagebreak