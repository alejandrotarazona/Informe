	\chapter{DESARROLLO DEL PROYECTO}

        \section{TÍTULO DEL PROYECTO}
            Alfabetización digital para obreros (proyecto piloto SUTES /USB AT-1202).
            
        \section{OBJETIVO GENERAL}
            Capacitar al obrero para el uso de tecnologías infromáticas.
            
        \section{OBJETIVOS ESPECÍFICOS}
            \begin{enumerate}
                \item Usar navegadores de internet
                \item Usar buscadores en internet
                \item Usar correo electrónico
                \item Usar banca electrónica
                \item Usar redes sociales
                \item Usar oficinas virtuales
                \item Usar responsablemente el internet
                \item Usar herramientas de ofimática
                \item Usar internet en autoaprendizaje
            \end{enumerate}
            
        \section{EJECUCIÓN DE ACTIVIDADES REALIZADAS}

             \subsection {Recibir capacitación en educación de adultos.}
             \begin{itemize}
                 \item Horas acretitables: 12
                 \item Período: 08 de diciembre al 01 de febrero
                \end{itemize}
             
             La capacitación fue dictada por la Lic. Evelyn Abdala (Coordinadora del Programa de Igualdad de
             Oportunidades).
             
             Entre los temas abordados en la capacitación se encuentran:
             
             \begin{itemize}
                 \item Preparación de las clases. Las clases deberían venir preparadas previamente y contar con tres fases bien definidas: inicio, intermedio y cierre.
                 
                 \item Técnicas grupales para mejorar la atención.
                 
                 \item Curvas de aprendizaje y relajación para lograr un mejor resultado.
                 
                 \item Manejo de grupos intranquilos, en cuyo caso se debería mantener un lenguaje formal y evitar desprestigiar al alumno.
                 
                \end{itemize}
             
             \subsection {Realizar entrevistas a los obreros}
             \begin{itemize}
                 \item Horas acretitables: 8
                 \item Período: 12 de enero al 19 de enero
                \end{itemize}
             
             En esta fase se dedicó a recabar información sobre los aspirantes al curso en los ámbitos socio-económico y académico para poder así llevar a cabo un diagnóstico de casos y una mejor agrupación de los mismos.
             
             Se contó con la colaboración del Sindicato Único de Trabajadores de la Educación Superior a travéz del sr. José Bermúdez, presidente del sindicato y representante de la comunidad.
             
             \subsection {Hacer diagnostico de cada participante}
             \begin{itemize}
                 \item Horas acretitables: 8
                 \item Período: 20 de enero al 24 de enero
                \end{itemize}
                
                Luego de haber recabado la información necesaria se procedió a realizar un análisis de cada una de las encuestas. Se obtuvo una clasificación de dos niveles de instrucción previa (niveles básico y avanzado) y se dividió a los grupos en sub grupos que contenían personas de nivel 1, niveles 1 y 2 juntos y nivel 2.
                
                El grupo de obreros tomado en cuenta para este informe se ubica en niveles 1 y 2, el mismo consta de:
                \begin{itemize}
                    \item Julio Camacho (Mantenimiento)
                    \item Noris Meza (Cocina)
                    \item María Peña (Cocina)
                    \item Ocarina Esparragoza (Mantenimiento)
                    \item Pedro Revete (Transporte)
                    \item Juan Gutiérrez
                    \item Dania Velázquez (Mensajería)
                    \item Hermes Aponte (Seguridad)
                    \item Gilberto Escalante (Cocina, inscrito luego del comienzo del curso)
                \end{itemize}
                
             \subsection {Elaborar planes de capacitación}
             \begin{itemize}
                 \item Horas acretitables: 4
                 \item Período: 24 y 25 de enero
                \end{itemize}
                
                Los preparadores hicieron la división de los temas según los niveles asignados (1, 1-2, 2), para así poder utilizar las guías del Plan Nacional de Alfabetización Tecnológica (PNAT), elaborado por la Fundación Infocentro en el año 2009.
                
                \begin{itemize}
                    \item Nivel 1:
                    \begin{itemize}
                        \item Modulo I: Iniciación al uso del computador.
                        \item Modulo V: Buscando y navegando para encontrar información al instante.
                    \end{itemize}
                    \item Nivel 2:
                    \begin{itemize}
                        \item Modulo II: Desarrollando nuestras ideas en un procesador de palabras.
                        \item Modulo III: Comunicando a través de presentaciones creativas con Impress.
                        \item Modulo IV: Calculando y graficando datos con la hoja de cálculo.
                        \item Modulo VI: Socializando a través de internet.
                    \end{itemize}
                \end{itemize}
                
            %\subsection {Elaboración de planes de capacitación}
            
            El plan de trabajo original (Nivel 1) al que se refiere este informe, sufrió una reprogramación dada la asistencia y el nivel de los alumnos, pasando a ser Nivel 2.
            
            El plan de trabajo elaborado para el grupo en cuestión, luego de la reprogramación, es el siguiente:
			\begin{center}
                
		
            \begin{tabular}{|c|p{10cm}|}
						\hline
                        \textbf{Clase} & \textbf{Temas a tratar} 
                        \\ \hline
                        1 & \raggedright Introducción y normas
                        \\ Presentación
                        \\ Temas a tocar durante el curso
                        \tabularnewline \hline
                        2 & \raggedright Definiendo Writer\cite{writer}
                        \\ Accediendo Writer
                        \\ Interactuando con la interfaz gráfica
                        \\ Comenzando a escribir textos y aplicar formatos
                        \tabularnewline \hline
                        3 & \raggedright Creando y guardando el documento 
                        \\ Abriendo un documento ya existente
                        \\ Configurando páginas
                        \\ Colocando márgenes
                        \tabularnewline \hline
                        
                        4 & \raggedright Actividad Práctica 1
                        \tabularnewline \hline
                        
                        5 & \raggedright Insertando imágenes
                        \\ Insertando tabla
                        \\ Insertando filas y columnas en la tabla
                        \tabularnewline \hline

                        6 & \raggedright Actividad Práctica 2
                        \tabularnewline \hline   
                    
                        7 & \raggedright Revisando la ortografía
                        \\ Observado nuestro documento antes de imprimir
                        \\ Vista preliminar
                        \tabularnewline \hline
                         
                        \end{tabular}     
                        
                        \begin{tabular}{|c|p{10cm}|}\hline 

                        8 & \raggedright Definiendo la hoja de cálculo Calc\cite{calc}
                        \\ Accediendo a la hoja de cálculo Calc
                        \\ Interactuando con la interfaz gráfica
                        \\ Creando y guardando la hoja de cálculo
                        \\ Abriendo la hoja de cálculo
                        \tabularnewline \hline
                                   
                        9 & \raggedright Actividad Práctica 3
                        \tabularnewline \hline
                        
                        10 & \raggedright Actividad Evaluada 1
                        \tabularnewline \hline
                        
                        11 & \raggedright Identificando una celda
                        \\ Seleccionando un rango de celda
                        \\ Seleccionando una fila o columna entera
                        \\ Combinando celdas
                        \\ Utilizando e identificando filas, columnas y celdas
                        \tabularnewline \hline
                        12 & \raggedright Para introducir fórmulas o funciones
                        \tabularnewline \hline
                        
                        13 & \raggedright Actividad Práctica 4
                        \tabularnewline \hline
                        
                        14 & \raggedright Insertando símbolos
                        \\ Insertando imágenes
                        \\ Modificando imágenes
                        \\ Insertando filtros
                        \\ Observando nuestro trabajo a través de la vista preliminar
                        \tabularnewline \hline
                        
                        15 & \raggedright Ordenando datos
                        \\ Actividad Práctica 5
                        \tabularnewline \hline
                        
                        16 & \raggedright Actividad Evaluada 2
                        \tabularnewline \hline

                        17 & \raggedright Definiendo el Gestor de Presentaciones Impress\cite{impress}
                        \\ Accediendo al Gestor de Presentaciones Impress
                        \\ Configurando la presentación a través del asistente
                        \\ Interactuando con la interfaz gráfica
                        \tabularnewline \hline
                         
                        \end{tabular}     
                        
                        \begin{tabular}{|c|p{10cm}|}\hline 
                        
                        18 & \raggedright Cuadro de diapositivas
                        \\ Panel de tareas
                        \\ Tipos de vistas de las diapositivas
                        \\ Editando texto
                        \\ Insertando cuadro de texto
                        \tabularnewline \hline
                        19 & \raggedright Modificando texto en una diapositiva
                        \\ Borrar texto de una diapositiva
                        \\ Insertar tablas (Hoja de Cálculo)
                        \tabularnewline \hline
                        
                        20 & \raggedright Actividad Práctica 6
                        \tabularnewline \hline
                        21 & \raggedright Actividad Evaluada 3
                        \tabularnewline \hline
                        
            \end{tabular}                                
            \end{center}	 
             \subsection {Preparar material de apoyo}
             \begin{itemize}
                 \item Horas acretitables: 16
                 \item Período: 25 de enero al 30 de marzo
                \end{itemize}
                
                Se elaboró material de apoyo, actividades y planes de tareas para los alumnos.
                
             \subsection {Dictar clases en cursos de capacitación}
             \begin{itemize}
                 \item Horas acretitables: 28
                 \item Período: 2 de febrero al 17 de abril
                \end{itemize}
                
                El inicio del curso fue precedido por la reunión general entre preparadores, obreros y estuvieron presentes el profesor Leonid Tineo por parte del grupo de profesores y el sr. José Bermúdez como representante de la comunidad.
                
                El curso al cual se refiere el presente informe fue dictado los días lunes y miércoles de 9:30 a 11:30 am y los días viernes de 1:30 a 3:30 pm.
                
             \subsection {Acompañar en el proceso de aprendizaje}
             \begin{itemize}
                 \item Horas acretitables: 28
                 \item Período: 2 de febrero al 17 de abril
                \end{itemize}
                
                Se realizó un seguimiento personalizado del progreso de cada uno de los alumnos participantes, evidenciado en las actividades prácticas.
                
             \subsection {Validar logros de los participantes}
             \begin{itemize}
                 \item Horas acretitables: 9
                 \item Período: 2 de febrero al 17 de abril
                \end{itemize}
                
                El preparador realizó la validación de los logros obtenidos por los participantes al finalizar cada módulo y el mismo está respaldado por las actividades evaluadas.
                
                Se presentaron diversos inconvenientes con los alumnos, desde problemas con la permisología hasta problemas familiares, de los 7 iniciales, el sr. Julio Camacho asistió a las clases y el sr. Gilberto Escalante, que se inscribió pasada una semana del comienzo del curso.
                
                \begin{enumerate}
                    \item Julio Camacho
                    
                    A pesar que  tenía conocimientos previos sobre el uso de correo electrónico, banca electrónica y  comercio electrónico, el sr Julio Camacho carecia de conocimientos sobre procesador de palabras, manejador de presentaciones y hojas de cálculo. Esto promovió su asistencia perfecta.
                   
                    El sr Julio culminó satisfactoriamente el curso, habiendo aprendido sobre los temas antes mencionados.
                    
                    \item Gilberto Escalante
                    
                    Desde su inclusión en el curso demostró gran interés y muy buen ánimo para el aprendizaje. Manejaba bien el correo electrónico y tenía conocimientos básicos de banca electrónica y redes sociales. Tuvo problemas con la asistencia debido a inconvenientes familiares acaecidos a lo largo del curso. Debido a ello, el sr Gilberto, no culminó el curso.
                    
                    \item Noris Meza
                    
                    No pudo asistir al curso por negativa de permisos por parte de su supervisor.
                    
                    \item María Peña
                    
                    No pudo asistir al curso por negativa de permisos por parte de su supervisor.
                    
                    \item Ocarina Esparragoza
                    
                    Abandonó el curso luego de dos clases.
                    
                    \item Pedro Revete
                    
                    Solicitó cambio de grupo.
                    
                    \item Juan Gutiérrez
                    
                    No se presentó al curso.
                    
                    \item Dania Velásquez
                    
                    No pudo asistir al curso por motivos familiares.
                    
                    \item Hermes Aponte
                    
                    No se presentó al curso.
                    
                \end{enumerate}
                
             \subsection {Elaboración de Informe Final}
             \begin{itemize}
                 \item Horas acretitables: 8
                 \item Período: 30 de marzo al 17 de abril
                \end{itemize}
                
                Se elaboró el presente informe con la descripción general del proyecto y un resumen de las actividades realizadas.

	\pagebreak
	
