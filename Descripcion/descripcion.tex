	\chapter{Descripción del problema}
    
		\section{Descripción de la Comunidad}
			Comunidad de obreros que laboran en la Universidad Simón Bolívar. \cite{planSC}
            \begin{itemize}
                \item Mujeres 120
                \item Hombres 150
                \item Discapacitados 70
                \item Edades comprendidas entre 18 y 60 años. 
                \item Niveles educativos que van desde primaria incompleta hasta educación superior incompleta, según lo recopilado en las encuestas recientes que realizó SUTES para la captación de participantes.
            \end{itemize}
            
            En general han tenido un acceso limitado o nulo a las tecnologías de información o no tienen acceso a un computador en casa, y pocos son los que tienen acceso a internet.
            
		\section{Antecedentes del Proyecto}
            Como primer antecedente se tiene el "Plan Nacional de Alfabetización Tecnológica" (PNAT)\cite{PNAT}, diseñado en 2009, el cual contempla la formación en ofimática desde el  uso del computador hasta el uso de redes sociales y banca electrónica\cite{infocentro}.
            
            Además están los períodos previos en los cuales se llevó a cabo este proyecto de servicio comunitario, éstos son:
            
            \begin{enumerate}
                \item \underline{Abril-Julio 2012:} 
                
                El primer grupo: fueron 4 preparadores, atendieron 24 trabajadores y finalizaron en fechas diferentes entre diciembre 2012 y marzo 2013.
                
                \item \underline{Sepitembre-Diciembre 2012:}
                
                El segundo grupo: inició en octubre de 2012, fueron 8 preparadores, atendieron 48 trabajadores y finalizaron en enero de 2013.
                
                \item \underline{Abril-Julio 2013:}
                
                Se prestó el servicio comunitario a un total de 48 personas, 29  Hombres y 19 Mujeres,  agrupados en 16 secciones, cada una conformada por grupos de 3 o 4 personas.
               
                \item \underline{Abril-Julio 2014}:
                
                El servicio comunitario se les dictó a un total de 49 obreros, 34 hombres y 15 mujeres, agrupados en 10 secciones, cada una conformada por 4 o 5 obreros
                
                %\item \underline{Enero-Marzo 2015:}
                
                %Iniciaron 6 preparadores pero hubo uno que por inconvenientes tuvo que retirar el servicio comunitario. Los demás culminaron satisfactoriamente.
                
                %En esta ocasión se inscribieron en el curso un total de 43 obreros, agrupados en 5 secciones de entre 7 y 10 obreros. Hubo 2 ingresos tardíos.
                
                %\underline{Inscritos:}
                %\begin{itemize}
                %    \item Total: 43 obreros.
                %    \item Hombres 27
                %    \item Mujeres 16
                %    \item 2 inscripciones tardías.
                %    \item 13 Nivel 1
                %    \item 6 Nivel 1-2
                %    \item 24 Nivel 2.
                %\end{itemize}
                
                %Se ubicó a más de la mitad por encima del nivel básico debido a que ya poseían los conocimientos básicos del computador y también habían participado en las ediciones previas del servicio comunitario.
                
                
            \end{enumerate}
            
	\pagebreak