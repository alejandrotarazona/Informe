	\section{DESCRIPCIÓN DEL PROBLEMA}
    
		\subsection{DESCRIPCIÓN DE LA COMUNIDAD}
			Comunidad de obreros que laboran en la Universidad Simón Bolívar. Edades comprendidas entre 18 y 60 años.
            
            En general han tenido un acceso limitado o nulo a las tecnologías de información, no tienen acceso a un computador en casa y pocos son los que tienen acceso a internet.
            
		\subsection{ANTECEDENTES DEL PROYECTO}
            Como principal antecedente se tiene el "Plan Nacional de Alfabetización Tecnológica" (PNAT), diseñado en 2009, el cual contempla la formación en ofimática desde el  uso del computador hasta el uso de redes sociales y banca electrónica.
            
            Además están los períodos previos en los cuales se llevó a cabo el curso de capacitación a los obreros, estos son:
            
            \begin{enumerate}
                \item \underline{Abril-Julio del año 2012:} 
                
                El primer grupo: fueron 4 preparadores, atendieron 24 trabajadores y finalizaron en fechas diferentes entre diciembre 2012 y marzo 2013.
                
                \item \underline{Sepitembre-Diciembre 12:}
                
                El segundo grupo: inició en octubre de 2012, fueron 8 preparadores, atendieron 48 trabajadores y finalizaron en enero de 2013.
                
                \item \underline{Abril-Julio 2013:}
                
                Se prestó el servicio comunitario a un total de 48 personas, 29  Hombres y 19 Mujeres,  agrupados en 16 secciones, cada una conformada por grupos de 3 o 4 personas.
               
                \item \underline{Abril-Jululio 14}:
                
                El servicio comunitario se les dictó a un total de 49 obreros, 34 hombres y 15 mujeres, agrupados en 10 secciones, cada una conformada por 4 o 5 obreros
                
                \item \underline{Enero-Marzo 2015:}
                
                Iniciaron 6 preparadores pero hubo uno que por inconvenientes tuvo que retirar el servicio comunitario. Los demás culminaron satisfactoriamente.
                
                En esta ocasión se inscribieron en el curso un total de 43 obreros, agrupados en 5 secciones de entre 7 y 10 obreros. Hubo además 3 ingresos tardíos.
            \end{enumerate}
            
	\pagebreak