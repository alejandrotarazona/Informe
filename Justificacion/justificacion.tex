\section{JUSTIFICACIÓN DEL SERVICIO COMUNITARIO}
		Muchos servicios e información actualmente sólo están disponibles en Internet. SUTES ha atendido personas desfavorecidas porque no supieron leer información enviada por Internet sobre lapsos de espera del HCM. También hay personas las cuales, teniendo que realizar informes o reportes en determinadas situaciones, por desconocimiento de las herramientas, procedían a realizarlos manualmente (manuscritos) o no los realizaban, cabe destacar que muchos de estos informes también se debían enviar por correo electrónico.
        
        Dada esta situación se pensó en incluir un el proyecto, siguendo los lineamientos del Plan Nacional de Alfabetización Tecnológica, en el cual se les brindara a los obreros la oportunidad de recibir los conocimientos necesarios para su inclusión en las nuevas tecnologías, desde el uso de un computador, pasando por el manejo de herramientas de oficina y terminando por el manejo de correos electrónicos, redes sociales y comercio electrónico. De esta forma no estarían excluídos de la información referente al sindicato y además, podrán gestionar mejor sus documentos y procedimientos regulares de su oficio.
        
        Además significa una mejora en la calidad de vida de los mismos ya que les permite mantener contacto con sus allegados mediante las redes sociales, realizar transferencias bancarias y pagos de servicios, dejándoles más tiempo que dedicar a sí mismos.
	\pagebreak