\chapter{JUSTIFICACIÓN DEL SERVICIO COMUNITARIO}
        En la actualidad la tecnología está permeando en todos los ámbitos de la vida cotidiana: transacciónes bancarias, recargas de saldo para telefonías, pagos de luz y agua, compras, todo esto ha ido migrando de la forma tradicional (con dinero en efectivo, tarjetas o chaques) a formas automatizadas mediante internet. La banca electrónica está en auge y con ello están también cambiando el modo de hacer transacciones comerciales.
        
        También tenemos las nuevas vías de transmisión de información dentro de las organizaciones empresariales que también han ido cambiando y las tradicionales carteleras informativas son un método que, si bien no será totalmente descartado, están siendo abandonados por las organizaciones. Casi cualquier empresa posee su propia página web y su servicio de correo electrónico.

		Particularmente tenemos el caso de SUTES (Sindicato Único de Trabajadores de la Educación Superior) quienes han estado cambiando sus métodos de difusión de información a una alternativa web, el correo electrónico, sin embargo la información no ha llegado a todos los destinatarios. SUTES ha tenido personas desfavorecidas porque no supieron leer información enviada por Internet sobre lapsos de espera del HCM. También hay personas las cuales, teniendo que realizar informes o reportes en determinadas situaciones, por desconocimiento de las herramientas, procedían a realizarlos manualmente (manuscritos) o no los realizaban, cabe destacar que muchos de estos informes también se debían enviar por correo electrónico. Tal es el caso de vigilancia, mantenimiento y transporte que deben hacer reportes regularmente sobre el acontecer diario y los respectivos departamentos quieren agilizar los procesos para llevar los reportes de manera electrónica; con este curso se les brinda la facilidad de ir a la par con los nuevos procedimientos (a los trabajadores) y forma parte de los cursos de capacitación que deben ser auspiciados por el ente empleador según la Ley Organica del Trabajo.
        
        Dada esta situación se pensó en desarrollar un el proyecto de alfabetización tecnológica, siguendo los lineamientos del Plan Nacional de Alfabetización Tecnológica\cite{infocentro,PNAT}, en el cual se les brindara al personal obrero de la Universidad Simón Bolívar la oportunidad de recibir los conocimientos necesarios para que tengan acceso a las nuevas tecnologías y puedan usarlas en la mejora de la produccividad y la calidad del trabajo; desde el uso de un computador, pasando por el manejo de herramientas de oficina y terminando por el manejo de correos electrónicos, redes sociales y comercio electrónico. De esta forma no estarían excluídos de la información referente al sindicato y además, podrán gestionar mejor los documentos y procedimientos regulares de su oficio.
        
        Además significa una mejora en la calidad de vida de los mismos ya que les permite mantener contacto con sus allegados mediante las redes sociales, realizar transferencias bancarias y pagos de servicios, dejándoles más tiempo que dedicar a sí mismos.
	\pagebreak