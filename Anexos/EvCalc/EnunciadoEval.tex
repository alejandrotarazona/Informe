Adjunto encontrará un archivo de Calc con dos (2) listas de precios de componentes de computadores y carros respectivamente.

Usando esa información usted deberá llenar los espacios vacíos sabiendo que:
\begin{enumerate}
    \item El computador que se quiere armar necesita tener
    \begin{itemize}
        \item 1 Tarjeta madre
        \item 1 Disco duro (al menos)
        \item 2 Memorias RAM
        \item 1 Teclado
        \item 1 Mouse
    \end{itemize}
    \item El carro necesita cambio de aceite y ciertos repuestos.
    \begin{itemize}
        \item 1 Alternador
        \item 5 Bujías
        \item 1 Batería
        \item 3 Cauchos
        \item 1 Bomba de gasolina
        \item 1 Bomba de frenos
        \item 1 Filtros de aceite
        \item 2 Pastillas de freno
        \item 2 Filtros de gasolina.
    \end{itemize}
    \item Como asignación extra (Celda G14) calcule el promedio de ambas compras.
\end{enumerate}